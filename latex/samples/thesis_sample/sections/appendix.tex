\chapter{Рекомендации по содержанию приложений}\label{app:A}

В приложения рекомендуется включать материалы, дополняющие текст ВКР, связанные с выполненной разработкой или исследованием, если они не могут быть включены в основную часть. Приложения могут включать: графический материал, таблицы не более формата А3, расчёты, описания алгоритмов и программ.

В приложения могут быть включены:
\begin{itemize}
  \item дополнительные материалы к ВКР или магистерской диссертации;
  \item промежуточные математические доказательства и расчеты;
  \item таблицы вспомогательных цифровых данных;
  \item протоколы экспериментов (испытаний);
  \item заключение метрологической экспертизы;
  \item инструкции, методики, описания алгоритмов и программ, разработанных в процессе выполнения ВКР;
  \item иллюстрации вспомогательного характера;
  \item копии технического задания на ВКР, программы работ или другие исходные документы для выполнения ВКР;
  \item протокол рассмотрения результатов выполненной разработки (исследования) на кафедре или научно-техническом семинаре кафедры;
  \item акты внедрения результатов ВКР или их копии;
  \item копии других документов, необходимых для оценки выполненной работы.
\end{itemize}



\chapter{Примеры вставки листингов программного кода}\label{app:B}

Для крупных листингов есть два способа. Первый с~подсветкой синтаксиса, 
но в~нём есть проблемы с поддержкой кириллицы, которая часто встречается 
в~комментариях и~текстовых строках. Он представлен на 
листинге~\ref{lst:nice}.
\begin{ListingEnv}[!h]% настройки floating аналогичны окружению figure
    \captiondelim{ } % разделитель идентификатора с номером от наименования
    \caption{Программа <<Hello, world>> на \protect\cpp}\label{lst:nice}
    % окружение учитывает пробелы и табуляции и применяет их в сответсвии с настройками
    \begin{lstlisting}[language={[ISO]C++}]
        #include <iostream>
        using namespace std;

        int main() //latin letters in comments
        {
            cout << "Hello, world" << endl; //comment
            system("pause");
            return 0;
        }
    \end{lstlisting}
\end{ListingEnv}%
Второй не~такой красивый, но без ограничений (см.~листинг~\ref{lst:plain}).
\begin{ListingEnv}[!h]
    \captiondelim{ } % разделитель идентификатора с номером от наименования
    \caption{Программа <<Hello, world>> без подсветки}\label{lst:plain}
    \begin{Verb}
        #include <iostream>
        using namespace std;

        int main() // комментарий на русском языке
        {
            cout << "Привет, мир" << endl;
        }
    \end{Verb}
\end{ListingEnv}

Первый способ рекомендуется использовать для вставки небольших фрагментов
кода внутри текста, а~второй "--- для вставки полного кода в~приложении.

Если нужно вставить короткий пример кода (одна или две строки),
то~выделение  линейками и нумерация может смотреться чересчур громоздко.
В таких случаях можно использовать окружения \texttt{lstlisting} или
\texttt{Verb} без \texttt{ListingEnv}. Приведём такой пример
с указанием языка программирования, отличного от~заданного по умолчанию:
\begin{lstlisting}[language=Python]
    reduce(lambda x, y: x * y, range(1, n+1))
\end{lstlisting}

Для оформления идентификаторов внутри строк
(функция \lstinline{main} и~т.\,п.) используется
\texttt{lstinline} или моноширинный текст
(\texttt{\textbackslash texttt}).

Далее приведён пример, когда Листинг~\ref{lst:external1} подгружается 
из внешнего файла. Здесь не используется дополнительное окружение, 
иначе код не переносится по страницам.
\begingroup
\captiondelim{ } % разделитель идентификатора с номером от наименования
\lstinputlisting[lastline=15,language={Python},caption={Листинг из внешнего файла},label={lst:external1}]{sources/code.py}
\endgroup





\chapter{Длинное название второго приложения, в котором приводится пример длинной таблицы}\label{app:C}

\section{Подраздел приложения}\label{app:C1}

Пример длинной таблицы с записью продолжения по ГОСТ 2.105:

\begin{longtblr}[
        theme=gost,
        caption = {Длинное длинное длинное длинное длинное длинное длинное название таблицы},
        entry = {Короткое название},
        label = {tab:long_table_test},
        remark{\so{Примечание}} = {Текст примечания. Текст примечания. Текст примечания. Текст примечания.},
    ]{
        colspec={|X[l]|c|c|X[l]|}, width=\textwidth,
        rowhead=1, rowfoot=1, rowsep=0pt, font=\small,
        row{1}={halign=c},
    }
    \hline
    Параметр & Тип & Умолч. & Описание \\ \hline
    \SetCell[c=4]{l}{Основные параметры} \\ \hline
    spark.driver.memory          & int & 1024 & определяет количество памяти, выделенной для драйвера Spark, в~мегабайтах \\
    spark.executor.memory        & int & 1024 & определяет количество памяти, выделенной для каждого из исполнителей Spark, в~мегабайтах \\
    spark.executor.cores         & int & 1    & определяет количество вычислительных ядер, выделяемых каждому исполнителю \\
    spark.executor.instances     & int & 1    & определяет количество исполнителей \\
    spark.driver .maxResultSize  & int & 1024 & определяет максимальный размер результата, который может быть передан от исполнителей (вычислительных узлов) обратно к драйверу Spark, в~мегабайтах \\
    spark.driver .memoryOverhead & int & 384  & определяет дополнительный объем памяти, выделяемый драйверу Spark для обеспечения более надёжной работы и~предотвращения переполнения памяти, в~мегабайтах \\
    \hline
    \SetCell[c=4]{l}{Дополнительные параметры} \\ \hline
    spark.dynamicAllocation .enabled & boolean & true & управляет активацией или деактивацией функции динамического выделения ресурсов \\
    spark.dynamicAllocation .minExecutors & int & 1    & определяет минимальное количество исполнителей, которое должно быть выделено в~кластере в~рамках динамического выделения ресурсов \\
    spark.dynamicAllocation .maxExecutors & int & $\infty$ & определяет максимальное количество исполнителей, которое может быть выделено в~кластере в~рамках динамического выделения ресурсов \\
    \hline
\end{longtblr}


\chapter{Свидетельство о государственной регистрации программы для ЭВМ}\label{app:D}
\noindent\includegraphics[width=0.98\textwidth]{svidetelstvo.jpg}
